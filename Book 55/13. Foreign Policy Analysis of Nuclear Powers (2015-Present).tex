\documentclass[a4paper, 11pt]{article}
\usepackage[margin=1in]{geometry}
\usepackage[utf8]{inputenc}
\usepackage[T1]{fontenc}
\usepackage{amsmath, amsfonts, amssymb}
\usepackage{setspace}

% Page setup
\onehalfspacing

% Document metadata
\title{\textbf{Foreign Policy Analysis of Nuclear Powers (2015-Present)}}
\author{Soumadeep Ghosh}
\date{Kolkata, India}

\begin{document}
\maketitle

\section{Analytical Framework}

This analysis employs multiple Foreign Policy Analysis (FPA) levels to examine the strategic behavior of nuclear powers:

\begin{itemize}
	\item \textbf{Individual Level}: Leadership characteristics and decision-making styles.
	\item \textbf{Domestic Level}: Bureaucratic politics, interest groups, and public opinion.
	\item \textbf{State Level}: National capabilities, geography, and institutional structures.
	\item \textbf{Systemic Level}: Balance of power, alliance structures, and international regimes.
\end{itemize}

\section{The Nine Nuclear Powers}

\subsection{United States}

\subsubsection{Leadership Transitions and Decision-Making}
The United States experienced significant foreign policy shifts across three administrations:
\begin{itemize}
	\item \textbf{Obama (2015-2017)}: Multilateral approach, nuclear modernization, Iran nuclear deal
	\item \textbf{Trump (2017-2021)}: "America First," withdrawal from international agreements, transactional diplomacy
	\item \textbf{Biden (2021-present)}: Alliance restoration, strategic competition framework
\end{itemize}

\subsubsection{Key Policy Shifts}
\begin{itemize}
	\item \textbf{Nuclear Doctrine}: Maintained nuclear triad modernization across administrations
	\item \textbf{Alliance Management}: Obama's pivot to Asia, Trump's burden-sharing emphasis, Biden's alliance revitalization
	\item \textbf{Strategic Competition}: Gradual shift from counter-terrorism to great power competition
\end{itemize}

\subsubsection{Domestic Influences}
Congressional oversight on nuclear policy, defense industrial complex lobbying, and public opinion shifts on international engagement have significantly shaped USA foreign policy decisions.

\subsection{Russia}

\subsubsection{Putin's Centralized Decision-Making}
Russian foreign policy demonstrates highly personalized decision-making centered on regime survival, with nuclear doctrine increasingly prominent in strategic messaging and escalatory nuclear rhetoric as a coercive tool.

\subsubsection{Major Policy Developments}
\begin{itemize}
	\item \textbf{2014-2015}: Ukraine crisis, annexation of Crimea
	\item \textbf{2015-2019}: Syrian intervention, strategic partnership with China
	\item \textbf{2022-present}: Full-scale invasion of Ukraine, nuclear threats
\end{itemize}

\subsubsection{Strategic Drivers}
Great power status restoration, NATO expansion concerns, and domestic legitimacy through nationalist foreign policy constitute the primary drivers of Russian strategic behavior.

\subsection{China}

\subsubsection{Xi Jinping's Leadership Style}
Xi Jinping's centralized decision-making, "core leader" status, long-term strategic planning, and "Chinese Dream" vision have replaced the previous "hide and bide" strategy with assertive diplomacy.

\subsubsection{Nuclear Policy Evolution}
\begin{itemize}
	\item Maintained "no first use" doctrine
	\item Significant nuclear arsenal expansion
	\item Integration of nuclear forces into broader A2/AD strategy
\end{itemize}

\subsubsection{Regional and Global Strategies}
Belt and Road Initiative expansion, South China Sea assertiveness, and strategic competition with the USA while avoiding direct confrontation characterize China's approach.

\subsection{United Kingdom}

\subsubsection{Brexit Impact on Foreign Policy}
The post-Brexit "Global Britain" strategy emphasizes renewed focus on Indo-Pacific engagement and nuclear deterrent modernization through Trident replacement.

\subsubsection{Alliance Relationships}
\begin{itemize}
	\item "Special relationship" with USA maintained
	\item AUKUS partnership formation (2021)
	\item European security commitments post-Brexit
\end{itemize}

\subsubsection{Strategic Adjustments}
Increased nuclear warhead cap (2021 Integrated Review), tilt toward Indo-Pacific region, and balancing European and global commitments represent key strategic shifts.

\subsection{France}

\subsubsection{Macron's Strategic Autonomy}
President Macron's "Europe that protects" vision emphasizes strategic autonomy while maintaining NATO membership and nuclear doctrine emphasizing deterrence credibility.

\subsubsection{Key Initiatives}
European defense integration advocacy, Sahel military interventions, and Indo-Pacific strategy development constitute France's primary strategic initiatives.

\subsubsection{Nuclear Posture}
France has maintained its independent nuclear deterrent, pursued modernization of nuclear forces, and engaged in nuclear sharing debates within EU context.

\subsection{India}

\subsubsection{Modi's Foreign Policy Transformation}
Prime Minister Modi's multi-alignment replacing non-alignment, Neighborhood First policy, and Act East policy expansion represent significant strategic shifts.

\subsubsection{Nuclear Developments}
\begin{itemize}
	\item Credible minimum deterrence maintenance
	\item Nuclear triad completion
	\item Nuclear submarine program advancement
\end{itemize}

\subsubsection{Strategic Partnerships}
Quad revitalization with USA, Japan, Australia, balancing China containment with economic interdependence, and Russia relationship management despite Western sanctions characterize India's approach.

\subsection{Pakistan}

\subsubsection{Civil-Military Relations Impact}
Military establishment's foreign policy influence, nuclear policy controlled by military, and China-Pakistan Economic Corridor (CPEC) as strategic anchor define Pakistan's strategic framework.

\subsubsection{Regional Security Complex}
India-centric security policy, Afghanistan policy complications, and balancing USA and China relations constitute Pakistan's regional challenges.

\subsubsection{Nuclear Strategy}
Full spectrum deterrence doctrine, tactical nuclear weapons development, and nuclear escalation management challenges characterize Pakistan's nuclear approach.

\subsection{Israel}

\subsubsection{Netanyahu's Long Tenure Impact (2015-2021, 2022-present)}
Iran-centric foreign policy, Abraham Accords breakthrough, and nuclear opacity doctrine maintenance define Israel's strategic approach.

\subsubsection{Strategic Priorities}
Iranian nuclear program prevention, regional alliance building, and USA relationship management across administrations constitute Israel's primary concerns.

\subsubsection{Nuclear Considerations}
Ambiguous nuclear posture maintained, preventive strike capabilities development, and regional nuclear balance concerns characterize Israel's nuclear strategy.

\subsection{North Korea}

\subsubsection{Kim Jong-un's Nuclear Diplomacy}
Nuclear weapons as regime survival tool, summit diplomacy with USA, South Korea, China, Russia, and nuclear program acceleration during diplomatic stalemates define North Korea's approach.

\subsubsection{Strategic Calculation}
Nuclear weapons as ultimate security guarantee, economic development vs. nuclear program trade-offs, and China dependency with autonomy preservation constitute North Korea's strategic calculus.

\section{Comparative Analysis}

\subsection{Decision-Making Patterns}

\subsubsection{Personalized vs. Institutionalized}
\begin{itemize}
	\item \textbf{High personalization}: Russia (Putin), China (Xi), North Korea (Kim)
	\item \textbf{Institutional constraints}: USA, U.K., France, India
	\item \textbf{Mixed systems}: Pakistan, Israel
\end{itemize}

\subsubsection{Nuclear Doctrine Evolution}
\begin{itemize}
	\item \textbf{Increased nuclear salience}: Russia, China, Pakistan
	\item \textbf{Modernization continuity}: USA, U.K., France
	\item \textbf{Capability development}: India, North Korea
	\item \textbf{Opacity maintenance}: Israel
\end{itemize}

\subsection{Systemic Pressures and Responses}

\subsubsection{Great Power Competition}
USA-China strategic competition drives regional alignments, Russia's revisionist behavior reshapes European security, and middle powers seek strategic autonomy.

\subsubsection{Alliance Dynamics}
NATO adaptation to hybrid threats and great power competition, Indo-Pacific partnerships emergence (Quad, AUKUS), and bilateral alliance strengthening vs. multilateral frameworks characterize contemporary alliance systems.

\subsubsection{Nuclear Governance Challenges}
Arms control regime erosion (INF Treaty withdrawal, New START extension), non-proliferation regime stress (Iran, North Korea), and emerging technologies impact (hypersonics, cyber, AI) present significant challenges.

\section{Key Trends and Implications}

\subsection{Nuclear Renaissance}
Increased nuclear weapon salience in security strategies, modernization programs across all nuclear powers, and nuclear coercion and signaling more prominent represent a significant shift.

\subsection{Alliance Reconfiguration}
Traditional alliances adapting to new threats, minilateral partnerships supplementing multilateral frameworks, and hedging strategies by middle powers characterize contemporary alliance systems.

\subsection{Technology and Doctrine}
Emerging technologies challenging traditional deterrence, nuclear-conventional integration increasing, and escalation management complexities growing present new strategic challenges.

\subsection{Domestic-International Linkages}
Leadership changes significantly impact foreign policy, domestic politics increasingly constraining international cooperation, and public opinion influence on nuclear policy growing demonstrate the importance of domestic factors.

\section{Conclusion}

The period from 2015-Present represents a significant shift in nuclear powers' foreign policies, characterized by:

\begin{itemize}
	\item \textbf{Strategic Competition Intensification}: USA-China rivalry and Russia's revisionism reshape global order.
	\item \textbf{Nuclear Weapons Renewed Salience}: After decades of marginalization, nuclear weapons regain prominence.
	\item \textbf{Alliance System Evolution}: Traditional frameworks adapt while new partnerships emerge.
	\item \textbf{Leadership-Driven Policies}: Personalized decision-making impacts strategic stability.
	\item \textbf{Technology-Doctrine Interaction}: Emerging technologies create new strategic challenges.
\end{itemize}

The Foreign Policy Analysis framework reveals how domestic factors, leadership characteristics, and systemic pressures interact to shape nuclear powers' strategic choices, with implications for global stability and nuclear governance. Understanding these complex interactions is crucial for policymakers and analysts seeking to navigate the evolving strategic landscape and maintain international security in an increasingly multi-polar nuclear world.

\end{document}